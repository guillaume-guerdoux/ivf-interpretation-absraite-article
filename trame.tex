\documentclass[french]{report}
\usepackage[utf8]{inputenc}

\begin{document}
\chapter{Thème : trame générique}

Indiquer les noms des contributeurs

\section*{Avant-propos}




\medskip
Afin d'avoir une présentation commune, les sources latex de ce chapitre vous seront communiquées (fichier trame.tex), ainsi que le fichier global (synthese.tex).

Pour ceux qui ne connaissent pas encore latex, la page

\begin{verbatim}
http://www.tuteurs.ens.fr/logiciels/latex/
\end{verbatim}

fournit une bonne introduction. Il n'est pas attendu que vous développiez une expertise particuliére en latex. Cependant,  vous étes invités é citer les  principales ressources que vous aurez utilisez en insérant par exemple \verb2\cite{0-Meyer14}2 pour référencer le livre de Bertrand Meyer, (ce qui donnera \cite{0-Meyer14} dans votre texte,) décrit dans un fichier .bib de la faéon suivante~:
\begin{verbatim}
@book{0-Meyer14,
  author    = {Bertrand Meyer},
  title     = {Agile! - The Good, the Hype and the Ugly},
  publisher = {Springer},
  year      = {2014}
}
\end{verbatim}

Pour ne pas avoir de clash entre vos fichiers,  vous étes invitez é précéder vos références par le numéro de votre théme (ici nous avons choisi le numéro 0).

\medskip
Les titres des sections suivantes sont donnés é titre indicatif : si l'une des sections n'est pas pertinente pour votre théme, vous pouvez la supprimer, la renommer, \ldots, é votre guise.

\medskip
Vous pouvez utiliser l'anglais ou le franéais.

\section{Introduction}

quelques mots qui introduisent le théme, situe le contexte, notamment les motivations, éléments historiques, \ldots

\section{Principe général}

description de quelques éléments techniques

\section{Exemple pédagogique}

illustration avec un exemple, petit si possible  (éventuellement seulement dans les grandes lignes)

\section{Outillage}

existence et caractériques éventuelles des outils en lien avec le théme

\section{Applications}

quelques pratiques ou applications dérivées

\section{Limitations}

ce qui n'est pas couvert par le théme

\end{document}

\documentclass[french]{report}
\usepackage[utf8]{inputenc}
\usepackage{babel}


\begin{document}
\chapter{Thème : Interprétation abstraite}

Mansouri Kasra
Tchakamian Romain
Guerdoux Guillaume

\section*{Avant-propos}




\medskip
Afin d'avoir une présentation commune, les sources latex de ce chapitre vous seront communiquées (fichier trame.tex), ainsi que le fichier global (synthese.tex).


fournit une bonne introduction.
Il n'est pas attendu que vous développiez une expertise particuliére en latex.
Cependant,  vous étes invités é citer les
principales ressources que vous aurez utilisez en
 insérant par exemple \verb2\cite{0-Meyer14}2 pour référencer l
 e livre de Bertrand Meyer, (ce qui donnera \cite{0-Meyer14} dans votre texte,)
  décrit dans un fichier .bib de la faéon suivante~:
\begin{verbatim}
@book{0-Meyer14,
  author    = {Bertrand Meyer},
  title     = {Agile! - The Good, the Hype and the Ugly},
  publisher = {Springer},
  year      = {2014}
}
\end{verbatim}

Pour ne pas avoir de clash entre vos fichiers,  vous étes invitez é précéder vos références par le numéro de votre théme (ici nous avons choisi le numéro 0).

\medskip
Les titres des sections suivantes sont donnés é titre indicatif : si l'une des sections n'est pas pertinente pour votre théme, vous pouvez la supprimer, la renommer, \ldots, é votre guise.

\medskip
Vous pouvez utiliser l'anglais ou le franéais.


\section{Introduction}

\paragraph{}
 TODO : ATTENTION, J'AI PAS MAL UTILISER UN SEUL DOCUMENT, RISQUE DE PLAGIAT
Nous assistons depuis plus de 25 ans à une explosion des performances du matériel
informatique, performances qui ont été multipliées par 10\up{4} à 10\up{6}. La taille des
programmes informatique a ainsi été accrue du même ordre de grandeur.
Malheureusement, les erreur et les bogues ont de la même façon une probabilité démultipliée
d'arriver et, au vu de la taille des nouveaux programmes, les méthodologies classiques de
révision du code sont tout simplement impossibles à mettre en oeuvre. \\

\paragraph{}
Une nouvelle idée a donc émergée, l'idée d'utiliser l'ordinateur lui-même pour détecter
les erreurs de programmation. Or ce travail est extrèmement complexe pour une machine dû à des
problèmes d'indécidabilité et de complexité. Nous ne pouvons donc nous résoudre seulement à une approximation
des comportements possibles du programme analysé.
Par exemple, le model checking fonctionne très bien sur des systèmes finis mais est
le plus souvent incomplet sur des systèmes infinis.  \\

Ces méthodes sont des cas particuliers d'interprétation abstraite de sémantique.

\section{Principe général}

\subsection{Rappel}
\paragraph{Modèle de calcul}
Un modèle de calcul est tout simplement une description mathématique formelle
de la succession des opérations effectuées par un programme sur un ordinateur ainsi que de
la modification des états internes de l'ordinateur et externes de l'environnement, et ceci
dans toutes les conditions possibles.

\paragraph{Sémantique}
Considérant un programme P, la sémantique de ce programme est une description des sens des
constructions du langage. En effet, similairement à un langage naturel, les phrases d'un programme
peuvent être syntaxiquement correctes mais ne pas toutes avoir un sens. Ainsi, en pratique,
la sémantique d'un programme P est un modèle de calcul qui décrit toutes les
éxécutions effective du programme P, et ce quel que soit l'environnement. \\
Il existe plusieurs types de sémantiques :
\begin{description}
    \item La sémantique axiomatique
    \item La sématique opérationnelle
    \item La sémantique dénotationnelle
\end{description}
Toutes ces sémantiques sont soit équivalentes, soit une approximation l'une de l'autre.

\paragraph{Spécification et vérification}
Pour un programme P, sa spécification est un modèle de calcul décrivant toutes ses
exécutions souhaitées (dans tous les environnements).  \\
Par exemple une spécification peut être tout simplement l'absence d'erreur arithmétique
dans le programme (division par 0). \\
Les spécifications peuvent être plus complexes et être données par des langages
de spécification plus précis (nottament les logiques temporelles). \\
Vérifier un programme P consiste à prouver qu’une sémantique de ce programme
satisfait une spécification S donnée. \\

Toutefois, de nombreuses questions relatives à la sémantique d'un programme P restent indécidables
comme la vérification du fait de la non calculabilité (en temps fini) des équations résultantes.

Il devient alors nécessaire de considérer des \textbf{approximations}.

\subsection{L'interprétation abstraite}
On définit comme sémantique concrète la sémantique la plus précise, celle qui décrit l'exécution
réelle d'un programme mais dont la vérification reste le plus souvent indécidables.
L’interprétation abstraite formalise l’idée qu’une sémantique est plus ou moins précise
selon le niveau d’observation auquel on se place. Ainsi, l'interprétation abstraite consiste à
déterminer des sémantiques liées par des relations d'abstractions.

\paragraph{Exemple illustratif}
On considère une boîte remplie de 100 billes de couleurs (jaunes, rouges et verts).
Chaque bille possède un ID, un numéro (non unique, entre 1 et 10 000)
et une couleur ((jaune, rouge ou vert).
Si nous souhaitions prouver que certaines billes n'étaient pas présentes, une manière serait
de tenir une liste comprenant le numéro et l'id de toutes les billes.
Mais nous aurions pu simplement récupérer leur numéro, si le numéro d'une bille est introuvable dans la liste,
alors celle-ci est absente. Toutefois, ce n'est pas parce que le numéro d'une bille est
trouvée dans la liste que celle-ci est dans la boîte (deux billes peuvent avoir le
même numéro). Toutefois, cette information, non absoule est suffisante dans la plupart
des cas (100 billes pouvant prendre des numéros entre 1 et 10 000, soit 1 chances sur 100 que deux billes aient
le même numéro).
Si nous ne sommes intéressés qu'à une information,
comme « Y a-t-il une bille jaune dans la boîte », il n'est
pas nécessaire de garder une liste de tous les numéros et couleurs de toutes les billets.
Nous pouvons, sans perdre de précision, nous
restreindre à maintenir une simple liste de la couleur des billes.

L'inteprétation abstraite, sur un exemple très simple, permet de rendre calculable, un cas qui ne l'est pas,
en prenant en compte le contexte ou en approximant la sémantique sémantique concrète.



\section{Exemple pédagogique}

illustration avec un exemple, petit si possible  (éventuellement seulement dans les grandes lignes)

\section{Outillage}

existence et caractériques éventuelles des outils en lien avec le théme

\section{Applications}

quelques pratiques ou applications dérivées

\section{Limitations}

ce qui n'est pas couvert par le théme

\end{document}

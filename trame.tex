\documentclass[french]{article}
\usepackage[utf8]{inputenc}
\usepackage{babel}
\usepackage{amsmath}
\usepackage[all]{xy}

\begin{document}

\title{Interprétation abstraite}

\author{Mansouri Kasra \\
        Tchakamian Romain \\
        Guerdoux Guillaume}

\maketitle


\section{Introduction}

\paragraph{}
Nous assistons depuis plus de 25 ans à une explosion des performances du matériel
informatique, performances qui ont été multipliées par 10\up{4} à 10\up{6}. La taille des
programmes informatiques a ainsi été accrue du même ordre de grandeur.
Malheureusement, les erreurs et les bogues ont de la même façon une probabilité démultipliée
d'arriver et, au vu de la taille des nouveaux programmes, les méthodologies classiques de
révision du code sont tout simplement impossibles à mettre en oeuvre. \\

\paragraph{}
Une nouvelle idée a donc émergé, l'idée d'utiliser l'ordinateur lui-même pour détecter
les erreurs de programmation. Or ce travail est extrèmement complexe pour une machine, dû à des
problèmes d'indécidabilité et de complexité. Nous ne pouvons donc nous résoudre seulement à une approximation
des comportements possibles du programme analysé.
Par exemple, le model checking fonctionne très bien sur des systèmes finis mais est
le plus souvent incomplet sur des systèmes infinis.  \\

\section{Principe général}

\subsection{Rappel}
\paragraph{Analyse statique et dynamique}
L'analyse statique consiste à obtenir des informations sur le comportement d'un programme
lors de son exécution sans réellement l'exécuter (en utilisant notamment des méthodes de
vérification formelle). \\
L'analyse dynamique consiste elle à suivre l’exécution d'un programme et d'en repérer les bogues.

Ainsi l'analyse statique se charge en amont de repérer des erreurs formelles de programmation
ou de conception, ainsi que la facilité de maintien du code.


\paragraph{Modèle de calcul}
Un modèle de calcul est tout simplement une description mathématique formelle
de la succession des opérations effectuées par un programme sur un ordinateur ainsi que de
la modification des états internes de l'ordinateur et externes de l'environnement, et ceci
dans toutes les conditions possibles.

\paragraph{Sémantique}
Considérant un programme P, la sémantique de ce programme est une description du sens des
constructions du langage. En effet, similairement à un langage naturel, les phrases d'un programme
peuvent être syntaxiquement correctes mais ne pas toutes avoir un sens. Ainsi, en pratique,
la sémantique d'un programme P est un modèle de calcul qui décrit toutes les
éxécutions effectives du programme P, et ce quel que soit l'environnement. \\
Il existe plusieurs types de sémantiques :
\begin{description}
    \item La sémantique axiomatique
    \item La sématique opérationnelle
    \item La sémantique dénotationnelle
\end{description}
Toutes ces sémantiques sont soit équivalentes, soit une approximation l'une de l'autre.

\paragraph{Spécification et vérification}
Pour un programme P, sa spécification est un modèle de calcul décrivant toutes ses
exécutions souhaitées (dans tous les environnements).  \\
Par exemple une spécification peut être tout simplement l'absence d'erreur arithmétique
dans le programme (division par 0). \\
Les spécifications peuvent être plus complexes et être données par des langages
de spécification plus précis (notamment les logiques temporelles). \\
Vérifier un programme P consiste à prouver qu’une sémantique de ce programme
satisfait une spécification S donnée. \\

Toutefois, de nombreuses questions relatives à la sémantique d'un programme P restent indécidables
comme la vérification du fait de la non calculabilité (en temps fini) des équations résultantes.

Il devient alors nécessaire de considérer des \textbf{approximations}.

\subsection{L'interprétation abstraite}
On définit comme sémantique concrète la sémantique la plus précise, celle qui décrit l'exécution
réelle d'un programme mais dont la vérification reste le plus souvent indécidables.
L’interprétation abstraite formalise l’idée qu’une sémantique est plus ou moins précise
selon le niveau d’observation auquel on se place. Ainsi, l'interprétation abstraite consiste à
déterminer des sémantiques liées par des relations d'abstractions. Elle peut être vue comme une sorte d'exécution partielle d'un programme informatique, pour avoir davantage d'informations sur sa sémantique (son flot de données par exemple), sans avoir à le traiter complètement. \\

Mahtématiquement cela reviendrait à choisir un domaine abstrait à partir d'un domaine concrèt donné. C'est-à-dire remplacer un ensemble d'objets S (états, traces, etc) par son abstraction $\beta(S)$. $\beta$ est appelée la fonction d'abstraction ou la fonction d'interprétation qui associe à chaque ensemble d'objets son domaine abstrait associé. La foncion inverse $\gamma$ est appelée la fonction de concrétisation, celle qui permet de mapper un ensemble d'objets abstraits vers des objets concrèts.

Une interprétation abstraite doit avoir les caractéristiques suivantes :
\begin{description}
    \item Etre solide : Aucune erreur ne doit être oubliée ( S $\subset \gamma(\beta(S))$ )
    \item Etre assez précise : Eviter les fausses alarmes
    \item Etre simple : être calculable en un temps raisonnable
\end{description}

 \begin{equation*}
  \xymatrix@C+6em@R+3em{
   S \ar[r]^{O} \ar[d]^{\beta} & S \ar[d]^{\beta} \\
    A \ar@{-->}[r]^{T} & A
  }
 \end{equation*}

 \begin{center}Figure 1. Illustration d'une abstraction \end{center}

\begin{description}
  \item S : Domaine concrèt
    \item A : Domaine abstrait
    \item O : S $\rightarrow$ S est la sémantique opérationnelle
    \item $\beta$ : S $\rightarrow$ A est la fonction d'abstraction
    \item T : A $\rightarrow$ A est l'abstraction de O ( fonction de transfert de A dans A)
\end{description}

Il est important de noter dans cette illustration que la transormation T conserve l'abstraction $\beta$. C'est à dire que si nous effectuons une abstraction $\beta$ puis appliquons la transformation T, nous obtenons le même résultat que si nous exécutions d'abord la sémantique opérationnelle O puis l'abstraction $\beta$. Cette propriété constitue l'un des principes fondamenteux sur lesquels nous nous basons pour faire de l'analyse de programme.

\section{Exemple pédagogique}

Nous allons faire une illustration de cette technique à l'aide d'un exemple simple et pédagogique. On considère une boîte remplie de 100 billes de couleurs (jaunes, rouges et verts).
Chaque bille possède un ID, un numéro (non unique, entre 1 et 10 000)
et une couleur ((jaune, rouge ou vert).
Si nous souhaitions prouver que certaines billes n'étaient pas présentes, une manière serait
de tenir une liste comprenant le numéro et l'id de toutes les billes.
Mais nous aurions pu simplement récupérer leur numéro, si le numéro d'une bille est introuvable dans la liste,
alors celle-ci est absente. Toutefois, ce n'est pas parce que le numéro d'une bille est
trouvée dans la liste que celle-ci est dans la boîte (deux billes peuvent avoir le
même numéro). Toutefois, cette information, non absoule est suffisante dans la plupart
des cas (100 billes pouvant prendre des numéros entre 1 et 10 000, soit 1 chances sur 100 que deux billes aient
le même numéro).
Si nous ne sommes intéressés qu'à une information,
comme « Y a-t-il une bille jaune dans la boîte », il n'est
pas nécessaire de garder une liste de tous les numéros et couleurs de toutes les billets.
Nous pouvons, sans perdre de précision, nous
restreindre à maintenir une simple liste de la couleur des billes.

L'inteprétation abstraite, sur un exemple très simple, permet de rendre calculable, un cas qui ne l'est pas,
en prenant en compte le contexte ou en approximant la sémantique sémantique concrète. \\

En informatique, nous pouvons prendre l'exemple d'un programme manipulant des variables entières. Une abstraction de sémantique pourrait par exemple consister à omettre la \textbf{valeur} des variables manipulées et uniquement garder leur \textbf{signe} pour représenter l'état du programme. Dans le cas de l'opération de multiplication, une telle abstraction reste précise car pour connaitre le signe d'un produit il suffit de connaitre le signe de ses opérandes. En revanche pour l'addition, cette abstraction perd en précision. Il est impossible de connaitre le signe d'une somme si ses opérandes sont de signe contraire.\\

En général, les abstractions de sémantique entrainent une perte de précision. Il y a donc un compromis à faire entre la précision et la faisabilité. On parle donc de \textbf{caclulabilité} vs \textbf{complexité}.


\section{Outillage: L'analyseur d'ASTREE}

\subsection{Introduction}
Les systèmes embarqués peuplent de plus en plus nos vies et sont de plus en plus complexes. Des vies humaines
dépendent de la fiabilité de ces systèmes embarqués (trains, voitures, avions, etc). Il est donc nécessaire de
s'assurer de la fiabilité de ces systèmes avant de les mettre en service.
C'est ce à quoi s'atelle l'analyseur d'ASTREE qui vise à prouver formellement
l'abscence de \textit{runtime errors} dans les programmes écrits en C. \\

Une \textit{runtime error} est une erreur dans un programme C violant la norme de C (par exemple un indice sortant
d'un tableau, ou une division par zéro), ou violant les directives de programmation (overflow), etc.

L'analyseur d'ASTREE est donc un analyseur de programmes basé sur l'interprétation abstraite. En effet, toutes les preuves
automatiques de programmes impliquent des formes d'approximation dans l'exécution du programme, formalisées par
l'interprétation abstraite.
Ainsi l'analyseur d'ASTREE, bien que limité aux \textit{runtime errors}, montre une première étape
dans la vérification formelle d'énormes systèmes embarqués par l'interprétation abstraite en un temps de calcul
raisonnable.

\subsection{L'analyseur ASTREE}

\paragraph{}
ASTREE possède plusieurs domaines d'applications. \\
Il est tout d'abord utilisé pour vérifier les programmes C embarqués dans les systèmes embarqués critiques
de systèmes complexes. En effet, il permet d'analyser les programmes C "classiques". Ceci sont composés de pointeurs,
tableaux, entiers, flotants, tests, boucles et appels à des fonctions, ceux sont les modèles qui sont embarqués dans les
systèmes critiques. Il n'est pas permis d'avoir des bouts de programmes "secondaires", c'est à dire des librairies C,
de l'allocation de mémoire dynamique et des fonctions récursives non bornées par exemples. L'analyseur d'ASTREE se
préocuppe donc d'analyser les programmes potentiellement embarquables dans des systèmes critiques. \\
De plus il vérifie les sémantiques opérationnelles concrètes (définies par la norme internationnal de C) qui
est instanciée par un comportement spécifique selon l'ordinateur et le compilateur et restaint aux utilisations faites par
l'utilisateur.
ASTREE vérifie aussi les sémantiques collectées qui sont un ensemble des traces partielles pour
la sémantique opérationelle concrète partant de l'état initial. Finalement, ASTREE vérifie la sémantique abstraite
qui se base sur les traces des états atteignables.

\paragraph{}
ASTREE est donc un analyseur de programme qui est :
\begin{description}
    \item Static : La vérification se fait en amont de l'exécution
    \item Automatique : Pas besoin d'adaptation par l'utilsateur selon le programme
    \item Basé sur la sémantique
    \item Solide : Il couvre tout l'espace des possibilités et ne peut laisser échapper une potentielle erreur
    \item Terminant : Il termine forcément
\end{description}
ASTREE est aussi \textit{multi-abstraction}, il n'utilise pas seulement qu'une unique forme d'abstraction mais réalise
un produit de nombreux domaines d'abstraction. Il met en relation les différentes analyses produites par les différentes
formes d'abstraction. L'abstraction d'ASTREE est dite spécialisable, car il peut inclure des nouvelles
formes d'abstraction ou ne pas utiliser certaines formes d'abstraction selon la catégorie du programme.


\subsection{Conclusion}
En conclusion de cette briève description d'ASTREE, nous pouvons déduire que cet analyseur est le pionnier de l'utilisation
d'interprétation abstraite pour vérifier des programmes en temps calculables. Pour 700 000 lignes
de codes, ASTREE ne met que 11h pour vérifier intégralement ce code.
Il se limite pour l'instant aux \textit{runtime errors} mais tout nous laisse imaginer qu'il
pourra un jour prendre en compte de nombreux autres types d'erreurs.


\section{Applications}

\paragraph{}
Aujourd'hui l'analyse statique de programme par interprétation abstraite est utilisée dans de nombreux secteurs industriels. En particulier pour les logiciels embarqués, ce genre d'approche est utilisée pour assurer la sûreté de logiciel. \\

Le premier exemple à citer est celui de l'analyse proposée dans "Theiling et al. 2000" qui a permis de calculer des bornes supérieurs au temps d'exécution de fragments de code assembleur, une tâche difficilement réalisable auparavant. Cette anaylse trouve son intérêt dans la vérification des logiciels temps réel. Par ailleurs l'analyseur Polyspace a vu le jour suite à l'échec du vol d'Ariane 5 pour analyser les application écrites avec le langage ADA. \\

Depuis 2002, l’analyseur statique ASTRÉE a été développé afin d’analyser avec précision et robustesse certaines familles d’applications critiques embarquées, et a effectivement
permis de prouver l’absence d’erreurs à l’exécution dans des applications avioniques
de taille industrielle. Par ailleurs, la recherche des sources d’imprécision dans les calculs flottants a également fait l’objet de travaux en analyse statique (Goubault et al., 2002). \\

Par exemple l'analyseur ASTRÉE a été appliqué au logiciel primaire de contrôle de vol du système de commande de vol électrique de a famille des Airbus A340 et A380. Les programmes étaient écrits en C et contenaient : \\
\begin{description}
  \item 100.000 à 250.000 lignes de code pour A340
    \item 400.000 à 1.000.000 lignes de code pour A380 \\
\end{description}

L'analyse d'ASTRÉE sur 400.000 lignes de code a donné les résultats suivants : \\
\begin{center}
\begin{tabular}{|l|c|r|}
  \hline
  temps & mémoire & fausse alarme \\
  \hline
  13h52 mn & 2,2 Go & 0 \\
  \hline
\end{tabular}
\end{center}

En se basant sur l'interprétation abstraite la communauté scientifque a réussi à formaliser des transformations de programmes telles que la compilation, la décompilation ainsi que le slicing. Ces formalismes permettent de combiner des analyses statiques et des transformations de programmes, par exemple, pour généraliser les algorithmes de transformation ou bien pour prouver des propriétés sur leur résultats.


\section{Références}
Voici la liste des références que nous avons utilisées pour rédiger cet article.
\begin{description}
    \item \textbf{Interprétation abstraite} : Article de Patrick Cousot du
    Département d’informatique de l'École normale supérieure.
    \item \textbf{Systematic Design of Program Transformation Frameworks by Abstract Interpretation} :
    Article de Patrick Cousot et de Radhia Cousot
    \item \textbf{The ASTREE Analyser} Article présentant l'analyseur d'Astrée, par le CNRS, l'Ecole normale supérieure
    et l'Ecole Polytechnique
    \item \textbf{Analyse Statique par Interprétation
Abstraite} article de Xavier Rival
  \item \textbf{Interprétation abstraite: application aux logiciels de l'A380} exposé de Patrick Cousot sur des questions d'actualité - académie des sciences
\end{description}
\end{document}
